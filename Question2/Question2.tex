\documentclass[11pt,preprint]{elsarticle}

\usepackage{lmodern}
%%%% My spacing
\usepackage{setspace}
\setstretch{1.2}
\DeclareMathSizes{12}{14}{10}{10}

% Wrap around which gives all figures included the [H] command, or places it "here". This can be tedious to code in Rmarkdown.
\usepackage{float}
\let\origfigure\figure
\let\endorigfigure\endfigure
\renewenvironment{figure}[1][2] {
    \expandafter\origfigure\expandafter[H]
} {
    \endorigfigure
}

\let\origtable\table
\let\endorigtable\endtable
\renewenvironment{table}[1][2] {
    \expandafter\origtable\expandafter[H]
} {
    \endorigtable
}


\usepackage{ifxetex,ifluatex}
\usepackage{fixltx2e} % provides \textsubscript
\ifnum 0\ifxetex 1\fi\ifluatex 1\fi=0 % if pdftex
  \usepackage[T1]{fontenc}
  \usepackage[utf8]{inputenc}
\else % if luatex or xelatex
  \ifxetex
    \usepackage{mathspec}
    \usepackage{xltxtra,xunicode}
  \else
    \usepackage{fontspec}
  \fi
  \defaultfontfeatures{Mapping=tex-text,Scale=MatchLowercase}
  \newcommand{\euro}{€}
\fi

\usepackage{amssymb, amsmath, amsthm, amsfonts}

\def\bibsection{\section*{References}} %%% Make "References" appear before bibliography


\usepackage[numbers]{natbib}

\usepackage{longtable}
\usepackage[margin=2.3cm,bottom=2cm,top=2.5cm, includefoot]{geometry}
\usepackage{fancyhdr}
\usepackage[bottom, hang, flushmargin]{footmisc}
\usepackage{graphicx}
\numberwithin{equation}{section}
\numberwithin{figure}{section}
\numberwithin{table}{section}
\setlength{\parindent}{0cm}
\setlength{\parskip}{1.3ex plus 0.5ex minus 0.3ex}
\usepackage{textcomp}
\renewcommand{\headrulewidth}{0.2pt}
\renewcommand{\footrulewidth}{0.3pt}

\usepackage{array}
\newcolumntype{x}[1]{>{\centering\arraybackslash\hspace{0pt}}p{#1}}

%%%%  Remove the "preprint submitted to" part. Don't worry about this either, it just looks better without it:
\makeatletter
\def\ps@pprintTitle{%
  \let\@oddhead\@empty
  \let\@evenhead\@empty
  \let\@oddfoot\@empty
  \let\@evenfoot\@oddfoot
}
\makeatother

 \def\tightlist{} % This allows for subbullets!

\usepackage{hyperref}
\hypersetup{breaklinks=true,
            bookmarks=true,
            colorlinks=true,
            citecolor=blue,
            urlcolor=blue,
            linkcolor=blue,
            pdfborder={0 0 0}}


% The following packages allow huxtable to work:
\usepackage{siunitx}
\usepackage{multirow}
\usepackage{hhline}
\usepackage{calc}
\usepackage{tabularx}
\usepackage{booktabs}
\usepackage{caption}


\newenvironment{columns}[1][]{}{}

\newenvironment{column}[1]{\begin{minipage}{#1}\ignorespaces}{%
\end{minipage}
\ifhmode\unskip\fi
\aftergroup\useignorespacesandallpars}

\def\useignorespacesandallpars#1\ignorespaces\fi{%
#1\fi\ignorespacesandallpars}

\makeatletter
\def\ignorespacesandallpars{%
  \@ifnextchar\par
    {\expandafter\ignorespacesandallpars\@gobble}%
    {}%
}
\makeatother


% definitions for citeproc citations
\NewDocumentCommand\citeproctext{}{}
\NewDocumentCommand\citeproc{mm}{%
\href{\#cite.\detokenize{#1}}{#2}\nocite{#1}}

\makeatletter
% allow citations to break across lines
\let\@cite@ofmt\@firstofone
% avoid brackets around text for \cite:
\def\@biblabel#1{}
\def\@cite#1#2{{#1\if@tempswa , #2\fi}}
\makeatother
\newlength{\cslhangindent}
\setlength{\cslhangindent}{1.5em}
\newlength{\csllabelwidth}
\setlength{\csllabelwidth}{3em}
\newenvironment{CSLReferences}[2] % #1 hanging-indent, #2 entry-spacing
{\begin{list}{}{%
	\setlength{\itemindent}{0pt}
	\setlength{\leftmargin}{0pt}
	\setlength{\parsep}{0pt}
	% turn on hanging indent if param 1 is 1
	\ifodd #1
	\setlength{\leftmargin}{\cslhangindent}
	\setlength{\itemindent}{-1\cslhangindent}
	\fi
	% set entry spacing
	\setlength{\itemsep}{#2\baselineskip}}}
{\end{list}}

\usepackage{calc}
\newcommand{\CSLBlock}[1]{\hfill\break\parbox[t]{\linewidth}{\strut\ignorespaces#1\strut}}
\newcommand{\CSLLeftMargin}[1]{\parbox[t]{\csllabelwidth}{\strut#1\strut}}
\newcommand{\CSLRightInline}[1]{\parbox[t]{\linewidth - \csllabelwidth}{\strut#1\strut}}
\newcommand{\CSLIndent}[1]{\hspace{\cslhangindent}#1}


\urlstyle{same}  % don't use monospace font for urls
\setlength{\parindent}{0pt}
\setlength{\parskip}{6pt plus 2pt minus 1pt}
\setlength{\emergencystretch}{3em}  % prevent overfull lines
\setcounter{secnumdepth}{5}

%%% Use protect on footnotes to avoid problems with footnotes in titles
\let\rmarkdownfootnote\footnote%
\def\footnote{\protect\rmarkdownfootnote}
\IfFileExists{upquote.sty}{\usepackage{upquote}}{}

%%% Include extra packages specified by user

%%% Hard setting column skips for reports - this ensures greater consistency and control over the length settings in the document.
%% page layout
%% paragraphs
\setlength{\baselineskip}{12pt plus 0pt minus 0pt}
\setlength{\parskip}{12pt plus 0pt minus 0pt}
\setlength{\parindent}{0pt plus 0pt minus 0pt}
%% floats
\setlength{\floatsep}{12pt plus 0 pt minus 0pt}
\setlength{\textfloatsep}{20pt plus 0pt minus 0pt}
\setlength{\intextsep}{14pt plus 0pt minus 0pt}
\setlength{\dbltextfloatsep}{20pt plus 0pt minus 0pt}
\setlength{\dblfloatsep}{14pt plus 0pt minus 0pt}
%% maths
\setlength{\abovedisplayskip}{12pt plus 0pt minus 0pt}
\setlength{\belowdisplayskip}{12pt plus 0pt minus 0pt}
%% lists
\setlength{\topsep}{10pt plus 0pt minus 0pt}
\setlength{\partopsep}{3pt plus 0pt minus 0pt}
\setlength{\itemsep}{5pt plus 0pt minus 0pt}
\setlength{\labelsep}{8mm plus 0mm minus 0mm}
\setlength{\parsep}{\the\parskip}
\setlength{\listparindent}{\the\parindent}
%% verbatim
\setlength{\fboxsep}{5pt plus 0pt minus 0pt}



\begin{document}



\begin{frontmatter}  %

\title{22894551 - Question 2 - Play that Song}

% Set to FALSE if wanting to remove title (for submission)




\author[Add1]{Charisa Amorie Geyer}
\ead{22894551@sun.ac.za}





\address[Add1]{Stellenbosch University, Cape Town, South Africa}



\vspace{1cm}





\vspace{0.5cm}

\end{frontmatter}

\setcounter{footnote}{0}



%________________________
% Header and Footers
%%%%%%%%%%%%%%%%%%%%%%%%%%%%%%%%%
\pagestyle{fancy}
\chead{}
\rhead{}
\lfoot{}
\rfoot{\footnotesize Page \thepage}
\lhead{}
%\rfoot{\footnotesize Page \thepage } % "e.g. Page 2"
\cfoot{}

%\setlength\headheight{30pt}
%%%%%%%%%%%%%%%%%%%%%%%%%%%%%%%%%
%________________________

\headsep 35pt % So that header does not go over title




\section{Introduction}\label{introduction}

This is an analysis of two iconic bands with \textgreater20-year
careers. Firstly, Coldplay (Alternative/Pop Rock, debut: 2000), and
secondly Metallica (Heavy Metal/Thrash, debut: 1981). Using Spotify
data, we can compare their musical progression, audience engagement, and
adaptation to industry trends, filtering only studio recordings for fair
comparisons.

\newpage

\section{Popularity by album}\label{popularity-by-album}

A box-and-whisker plot compares the popularity scores of each band's
albums, revealing trends across their discographies.

\subsection{Coldplay}\label{coldplay}

Coldplay has released a wide range of albums over the years. Coldplay's
evolving sound over the years is reflected in the varying popularity of
their studio and live albums.

\subsubsection{Overall}\label{overall}

\includegraphics{Question2_files/figure-latex/unnamed-chunk-1-1.pdf}
Coldplay is known for its live albums, with the live albums of ``Live in
Bueonos Aires'' and ``Live 2012'' surpassing albums like ``Love in
Tokyo'' and ``Everyday Life'' in popularity.

\newpage

\subsubsection{Live}\label{live}

\includegraphics{Question2_files/figure-latex/unnamed-chunk-2-1.pdf}
Live albums consistently rank highly, suggesting stronger fan engagement
with concert performances.

\newpage

\subsection{Metallica}\label{metallica}

Metallica's enduring appeal is evident, with remastered editions
dominating their most popular works.

\subsubsection{Overall}\label{overall-1}

\includegraphics{Question2_files/figure-latex/unnamed-chunk-3-1.pdf}
Remastered albums claim 7 of the top 10 spots, highlighting their
continued relevance and enhanced production value.

\newpage

\subsubsection{Studio}\label{studio}

\includegraphics{Question2_files/figure-latex/unnamed-chunk-4-1.pdf}
Classic releases maintain a strong presence, showing a loyal fanbase.
Streaming services have made Metallica's music much more accessible.
Seeing an steady increase in popularity over time is unsurprising.

\newpage

\section{Song popularity}\label{song-popularity}

Now let's look deeper into the popularity of the songs within each
album.

\begin{center}\includegraphics[width=0.8\linewidth]{Question2_files/figure-latex/unnamed-chunk-5-1} \end{center}

``Parachutes'' and ``A Rush of Blood to the Head'' are identified as
Coldplay's two most popular albums of all time. Within those two albums,
all songs are popular. Compared to ``Mylo Xyoto'', where many songs are
popular, but some are unpopular - bumping the album to a lower ranking
overall. This shows the importance of song popularity in determining
album popularity.

\newpage

\section{Timetrends in music style}\label{timetrends-in-music-style}

To look deeper into the longevity of these popular bands, we can
consider the development of music styles over time. By reviewing the
time trends of components of their music compared to that of Spotify (a
proxy for global music trends), we can uncover whether the artists
changed their style to meet global trends. Metallica, known as a loud
band, remains louder than the global trend, up until about 2015,
maintaining their signature style.
\includegraphics{Question2_files/figure-latex/unnamed-chunk-6-1.pdf}
Coldplay starts following the global danceability trends, but returns to
their calmer style over time.

\includegraphics{Question2_files/figure-latex/unnamed-chunk-7-1.pdf}

\newpage

\section{Live vs Studio albums}\label{live-vs-studio-albums}

We can see that energy trends are much lower in Live albums, for both
Coldplay and Metallica, but especially for Metallica.

\begin{center}\includegraphics[width=0.8\linewidth]{Question2_files/figure-latex/unnamed-chunk-8-1} \end{center}

\begin{center}\includegraphics[width=0.8\linewidth]{Question2_files/figure-latex/unnamed-chunk-8-2} \end{center}

\section{Conclusion}\label{conclusion}

Both Metallica and Coldplay demonstrate how iconic artists balance
staying true to their roots while navigating evolving music trends.
Ultimately, their sustained popularity suggests that success lies not in
chasing trends, but in blending selective innovation with core artistic
identity.

\bibliography{Tex/ref}





\end{document}
